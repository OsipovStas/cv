

%----------------------------------------------------------------------------------------
%	PACKAGES AND OTHER DOCUMENT CONFIGURATIONS
%----------------------------------------------------------------------------------------

\documentclass[11pt,a4paper,sans]{moderncv} % Font sizes: 10, 11, or 12; paper sizes: a4paper, letterpaper, a5paper, legalpaper, executivepaper or landscape; font families: sans or roman

\moderncvstyle{classic} % CV theme - options include: 'casual' (default), 'classic', 'oldstyle' and 'banking'
\moderncvcolor{grey} % CV color - options include: 'blue' (default), 'orange', 'green', 'red', 'purple', 'grey' and 'black'

\usepackage[scale=0.75]{geometry} % Reduce document margins
%\setlength{\hintscolumnwidth}{3cm} % Uncomment to change the width of the dates column
%\setlength{\makecvtitlenamewidth}{10cm} % For the 'classic' style, uncomment to adjust the width of the space allocated to your name
\usepackage{stackengine}
\def\Ruble{\stackengine{.64ex}{%
  \stackengine{.4ex}{\textbf{\textsf{P}}}{\rule{1ex}{.16ex}\kern.55ex}{O}{r}{F}{F}{L}%
  }{\rule{1ex}{.16ex}\kern.55ex}{O}{r}{F}{F}{L}\kern-.1ex}

\newcommand\Chref[3][cyan]{\href{#2}{\small\color{#1}#3}}

%----------------------------------------------------------------------------------------
%	NAME AND CONTACT INFORMATION SECTION
%----------------------------------------------------------------------------------------

\firstname{Stanislav} % Your first name
\familyname{Osipov} % Your last name

% All information in this block is optional, comment out any lines you don't need
\address{21/2 Komendantsky av.}{St.-Petersburg, Russian Federation}
\mobile{(911) 284 4911}
\email{stasstels@gmail.com}
\homepage{https://github.com/OsipovStas}{OsipovStas} % The first argument is the url for the clickable link, the second argument is the url displayed in the template - this allows special characters to be displayed such as the tilde in this example


%----------------------------------------------------------------------------------------

\begin{document}

\makecvtitle % Print the CV title
%----------------------------------------------------------------------------------------
%	GOAL SECTION
%----------------------------------------------------------------------------------------

\section{Goal}
\cvitem{Position}{Java Developer}{}{}{}{}  % Arguments not required can be left empty
%\cvitem{Salary}{100.000 \Ruble}{}{}{}{}

%----------------------------------------------------------------------------------------
%	ABOUT SECTION
%----------------------------------------------------------------------------------------

\section{About}
Passionate developer with solid background in math and good knowledge of algorithms and data structures. Willing to deliver clean and testable code. Enjoy to found smart solutions of complex tasks. Eager to progress, self-motivated and communicable. Interested in various areas of computer science. Particularly in Java-ecosystem, functional programming and software engineering.

%----------------------------------------------------------------------------------------
%	WORK EXPERIENCE SECTION
%----------------------------------------------------------------------------------------

\section{Experience}

\cventry{August 2014 -- Present}{Java Developer}{\textsc{\Chref{http://sidenis.ru/}{Sidenis}}}{Saint-Petersburg}{}{
Participating in development of various projects for \Chref{http://www.swissre.com/}{Swiss Reinsurance company}.
\newline{}
Projects:
\begin{itemize}
\item{\textit{Reference Data Service}}\newline{}
My  responsibilities are maintenance and developing one of core subsystem within SwissRe IT landscape. Reference Data Service has more than 40 client applications which use it everyday. Primary objectives are continuous delivery of a service, implementing new features and refactoring of existent code.
\newline{}
Technologies:
\begin{itemize}
\item{J2EE(EJB, JAX-WS), Hibernate, Spring IoC, Oracle, Maven, git}
\end{itemize}
\item{\textit{\Chref{http://www.swissre.com/clients/client_tools/about_puma.html}{Project Underwriting Management Application}}}\newline{}
It is a tool supporting Swiss Re's Engineering Underwriters in their underwriting process for construction and erection risks
\newline{}
Technologies:
\begin{itemize}
\item{J2SE, Swing, Firebird, Guice, Guava}
\end{itemize}
\end{itemize}
%Detailed achievements:
%\begin{itemize}
%\item Learned how to make amazing coffee
%\item Finally determined the reason for \textsc{PC LOAD LETTER}:
%\begin{itemize}
%\item Paper jam
%\item Software issues:
%\begin{itemize}
%\item Word not sending the correct data to printer
%\item Windows trying to print in letter format
%\end{itemize}
%\item Coffee spilled inside printer
%\end{itemize}
%\item Broke the office record for number of kitten pictures in cubicle
%\end{itemize}
}

%------------------------------------------------

\cventry{Summer 2013}{Intern developer}{\textsc{\Chref{https://www.jetbrains.com/}{Jet Brains}}}{Saint-Petersburg}{}{\textit{Responsibilities:} 
\newline{}
Developing refactoring tools for IntelliJ IDEA \Chref{https://plugins.jetbrains.com/plugin/?id=4050}{La Clojure plugin}
\newline{}
\newline{}
\textit{Results:}
\begin{itemize}
\item Implemented a handful of common refactorings, integrated with IntelliJ Platform
\item Created a simplified integration layer for adding more refactorings in the future
\item Clojure language has been used to write refactorings
\end{itemize}}

%------------------------------------------------
%----------------------------------------------------------------------------------------
%	SKILLS SECTION
%----------------------------------------------------------------------------------------



\section{Skills}
\cvitem{Programming languages}{\begin{itemize}
\item{Main:}
\begin{itemize}
\item {Java, Clojure, Scala}
\end{itemize}
\item{Secondary:}
\begin{itemize}
\item {C\textbackslash C++, Python, Haskell}
\end{itemize}
\end{itemize}}
\cvitem{IDE}{IntelliJ IDEA, PyCharm, Vim}
\cvitem{CI}{Atlassian JIRA, Bamboo}
\cvitem{Build Tools}{Maven, Ant}
\cvitem{DBMS}{Oracle, MySQL, Firebird, H2SQL}
\cvitem{Servers}{IBM WebSphere, Tomcat}
\cvitem{OS}{Unix-like, Windows}
\cvitem{VCS}{Git, SVN, CVS}
\cvitem{Others}{Guava, JUnit, Mockito, Java Concurrency, Hibernate, \newline{}Spring IoC, Guice, IntelliJ Platform Experience,\newline{} OpenCV, Intel TBB}


%----------------------------------------------------------------------------------------
%	EDUCATION SECTION
%----------------------------------------------------------------------------------------

\section{Education}

\cventry{2014}{M.Sc. in Computer Science}{\Chref{http://mit.spbau.ru/}{St.-Petersburg Academic University of the Russian
Academy of Science}}{}{}{Thesis:\newline{}
"Macroextensions for IntelliJ Scala Plugin"\newline{}
Advisor:\newline{}
Alexander Podkhalyuzin, Scala Plugin Team Lead\newline{}
Description:\newline{}
Scala Macros can generate additional members in type definitions during the
compilation. It makes the process of creation developing tools for Scala more
complicated. A solution to this problem is suggested in my diploma. A system
of lightweight IDE macroextensions is presented, which can dynamically upload
information about compiler generated methods into IDE.}  % Arguments not required can be left empty
\cventry{2011}{B.Sc. in Computer Science}{St.-Petersburg NRU of Information Technologies,
Mechanics and Optics(\Chref{http://www.ifmo.ru/}{ITMO})}{}{}{Thesis:\newline{}
"Model of tunnelling through periodic array of quantum dots in a magnetic field"\newline{}
Advisor:\newline{}
Prof. I. Yu. Popov\newline{}
Publication:\newline{}
Model of tunnelling through periodic array of quantum dots in a magnetic field\newline{}
I. Yu. Popov and S. A. Osipov, 2012 \Chref{http://iopscience.iop.org/article/10.1088/1674-1056/21/11/117306/meta;jsessionid=235552A24DC31E42C8C2DF02777357BF.ip-10-40-2-121}{Chinese Phys. B 21.}}

\subsection{Additional}
\cvitem{2012-2014}{\Chref{https://compscicenter.ru/}{Computer Science Center} student}
\cvitem{Courses and conferences}{\begin{itemize}
\item {\Chref{https://www.coursera.org/maestro/api/certificate/get_certificate?course_id=972351}{FP principles in Scala}}
\item {Joker 2014 and 2015 participant}
\item {\Chref{http://jokerconf.com/trainings/krivosheev/}{Refactoring 2.0 training}}
\end{itemize}}





%----------------------------------------------------------------------------------------
%	LANGUAGE SECTION
%----------------------------------------------------------------------------------------

\section{Language}

\cvitem{Russian}{Native}
\cvitem{English}{Upper-Intermediate}
\cvitem{German}{Basic knowledge}


\end{document}